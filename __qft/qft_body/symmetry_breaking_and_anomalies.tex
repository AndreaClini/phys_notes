\chapter{Symmetry Breaking \& Anomalies}

First of let us very intuitively explain the difference between the two concepts:
\begin{itemize}
    \item Symmetry breaking is when a symmetry of the \textit{Lagrangian} is not a symmetry of the \textit{ground state} (vacuum).
    For example, in the case of spontaneous symmetry breaking, the Lagrangian is symmetric under some group $G$, but the vacuum only under a subgroup $H \subset G$.
    Consider the case of the Mexican hat potential: the potential is symmetric under rotations around the vertical axis, but the ground state (the bottom of the hat) is not.
    On the other hand, the state sitting on top of the hat-- a fake vacuum-- does respect the full symmetry and is invariant under rotations.
    \item An anomaly is when a symmetry of the \textit{classical action} (i.e. of the classical Lagrangian) is not a symmetry of the \textit{quantum action} (i.e. of the quantum Lagrangian).
\end{itemize}