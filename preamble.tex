% ==================================================================
%====================== README (in progress)======================
%===================================================================
% PREAMBLE CONTENTS:
%----- Conventions
%----- Drafting and Collaborating
%----- General Setup
%----- Math, Graphs and Images
%----- Coding

%================================================================
%================ CONVENTIONS (in progress) ====================
%===============================================================
% 1. TAGS
% See the section DRAFTING and COLLABORATIONS below
% 2. LABELLING
% Equations, figures, tables etc are labelled in snake_case according
% to \label{eq:name_of_equation}, \label{fig:name_of_equation}, %\label{table:name_of_equation}
%


%==================================================================
%-------------------- DRAFTING and COLLABORATING  -------------------
%==================================================================

%==================================================
% ---------------Custom Tags------------------------
%    \citemiss              -- for mising citation(s)
%    \refmiss               --for missing reference to some label (eqref etc)
%   \notetag{...}           -- margin note (yellow)
%   \questiontag{...}       -- margin question (orange)
%   \todotag{...}           -- margin TODO (red)
%   \NoteInline{...}     -- inline note (yellow)
%   \QuestionInline{...} -- inline question (orange)
%   \TodoInline{...}     -- inline TODO (red)
%   \cite
%   \ref
%
% ---------------Color Conventions------------------
%   NOTE:      pale yellow     (NoteColor)
%   QUESTION:  light orange    (QuestionColor)
%   TODO:      pale red        (TodoColor)
%====================================================

\usepackage{comment} % for \begin{comment} environment
\usepackage{todonotes}
\usepackage{xspace}
\setlength{\marginparwidth}{1.5cm} %reduce to avoid overflow of margin tags

\definecolor{NoteColor}{RGB}{255,255,170}
\definecolor{QuestionColor}{RGB}{255,220,150}
\definecolor{TodoColor}{RGB}{255,170,170}
\definecolor{AnswerColor}{RGB}{100,255,100}

%---------------- missing \ref{} or \cite{} ------------------
\newcommand{\refmiss}{%
  {\setlength{\fboxsep}{1pt}%
  \fcolorbox{black}{TodoColor}{\scriptsize (*ref)}}\xspace
}

\newcommand{\citemiss}{%
  {\setlength{\fboxsep}{1pt}%
  \fcolorbox{black}{TodoColor}{\scriptsize (cite)}}\xspace
}

%-------------------- other tags: note, question, todo ----------------------
\newcommand{\notetag}[1]{%
  {\setlength{\fboxsep}{1pt}%
   \todo[backgroundcolor=NoteColor, bordercolor=black, linecolor=black, size=\scriptsize]{#1}}%
}
\newcommand{\questiontag}[1]{%
  {\setlength{\fboxsep}{1pt}%
   \todo[backgroundcolor=QuestionColor, bordercolor=black, linecolor=black, size=\scriptsize]{#1}}%
}
\newcommand{\todotag}[1]{%
  {\setlength{\fboxsep}{1pt}%
   \todo[backgroundcolor=TodoColor, bordercolor=black, linecolor=black, size=\scriptsize]{#1}}%
}
\newcommand{\answertag}[1]{%
  {\setlength{\fboxsep}{1pt}%
   \todo[backgroundcolor=AnswerColor, bordercolor=black, linecolor=black, size=\scriptsize]{#1}}%
}
\newcommand{\noteinline}[1]{%
  {\setlength{\fboxsep}{1.5pt}%
  \fcolorbox{black}{NoteColor}{\scriptsize \textcolor{black}{#1}}}%
}
\newcommand{\questioninline}[1]{%
  {\setlength{\fboxsep}{1.5pt}%
  \fcolorbox{black}{QuestionColor}{\scriptsize \textcolor{black}{#1}}}%
}
\newcommand{\todoinline}[1]{%
  {\setlength{\fboxsep}{1.5pt}%
  \fcolorbox{black}{TodoColor}{\scriptsize \textcolor{black}{#1}}}%
}
\newcommand{\answerinline}[1]{%
  {\setlength{\fboxsep}{1.5pt}%
  \fcolorbox{black}{AnswerColor}{\scriptsize \textcolor{black}{#1}}}%
}

%------------ tags of Andrea----------------------------
\newcommand{\noteAndrea}[1]{%
  {\setlength{\fboxsep}{1pt}%
   \todo[backgroundcolor=NoteColor, bordercolor=black, linecolor=black, size=\scriptsize]{\textbf{A}: #1}}%
}
\newcommand{\questionAndrea}[1]{%
  {\setlength{\fboxsep}{1pt}%
   \todo[backgroundcolor=QuestionColor, bordercolor=black, linecolor=black, size=\scriptsize]{\textbf{A}: #1}}%
}
\newcommand{\todoAndrea}[1]{%
  {\setlength{\fboxsep}{1pt}%
   \todo[backgroundcolor=TodoColor, bordercolor=black, linecolor=black, size=\scriptsize]{\textbf{A}: #1}}%
}
\newcommand{\answerAndrea}[1]{%
  {\setlength{\fboxsep}{1pt}%
   \todo[backgroundcolor=AnswerColor, bordercolor=black, linecolor=black, size=\scriptsize]{\textbf{A}: #1}}%
}
\newcommand{\noteinlineAndrea}[1]{%
  {\setlength{\fboxsep}{1.5pt}%
  \fcolorbox{black}{NoteColor}{\scriptsize \textcolor{black}{A:~#1}}}%
}
\newcommand{\questioninlineAndrea}[1]{%
  {\setlength{\fboxsep}{1.5pt}%
  \fcolorbox{black}{QuestionColor}{\scriptsize \textcolor{black}{A:~#1}}}%
}
\newcommand{\todoinlineAndrea}[1]{%
  {\setlength{\fboxsep}{1.5pt}%
  \fcolorbox{black}{TodoColor}{\scriptsize \textcolor{black}{A:~#1}}}%
}
\newcommand{\answerinlineAndrea}[1]{%
  {\setlength{\fboxsep}{1.5pt}%
  \fcolorbox{black}{AnswerColor}{\scriptsize \textcolor{black}{A:~#1}}}%
}

%----------------tags of Emanuele --------------------------
\newcommand{\noteEmanuele}[1]{%
  {\setlength{\fboxsep}{1pt}%
   \todo[backgroundcolor=NoteColor, bordercolor=black, linecolor=black, size=\scriptsize]{\textbf{E}: #1}}%
}
\newcommand{\questionEmanuele}[1]{%
  {\setlength{\fboxsep}{1pt}%
   \todo[backgroundcolor=QuestionColor, bordercolor=black, linecolor=black, size=\scriptsize]{\textbf{E}: #1}}%
}
\newcommand{\todoEmanuele}[1]{%
  {\setlength{\fboxsep}{1pt}%
   \todo[backgroundcolor=TodoColor, bordercolor=black, linecolor=black, size=\scriptsize]{\textbf{E}: #1}}%
}
\newcommand{\answerEmanuele}[1]{%
  {\setlength{\fboxsep}{1pt}%
   \todo[backgroundcolor=AnswerColor, bordercolor=black, linecolor=black, size=\scriptsize]{\textbf{E}: #1}}%
}
\newcommand{\noteinlineEmanuele}[1]{%
  {\setlength{\fboxsep}{1.5pt}%
  \fcolorbox{black}{NoteColor}{\scriptsize \textcolor{black}{E:~#1}}}%
}
\newcommand{\questioninlineEmanuele}[1]{%
  {\setlength{\fboxsep}{1.5pt}%
  \fcolorbox{black}{QuestionColor}{\scriptsize \textcolor{black}{E:~#1}}}%
}
\newcommand{\todoinlineEmanuele}[1]{%
  {\setlength{\fboxsep}{1.5pt}%
  \fcolorbox{black}{TodoColor}{\scriptsize \textcolor{black}{E:~#1}}}
}
\newcommand{\answerinlineEmanuele}[1]{%
  {\setlength{\fboxsep}{1.5pt}%
  \fcolorbox{black}{AnswerColor}{\scriptsize \textcolor{black}{E:~#1}}}%
}


%==================================================================
%-------------------- GENERAL SETUP -------------------
%==================================================================

%%%%%%%%%%%%%%%%%%%%%%%%%%%%%%%%%%%%%%%%%%%%%%%%%%%%%%%%%%%%%%%%%%%%%%%%%%%%%%%%%%%%
%%%%%%%%%%%%%%%%%%%%%%%%%%%%%%%%%%%%%%%%%%%%%%%%%%%%%%%%%%%%%%%%%%%%%%%%%%%%%%%%%
%-------------------- file extension (pdf variants etc for digital archiving)--------

% \pdfminorversion=4  % PDF 1.4 (required by PDF/A-1)


% \usepackage[a-1b]{pdfx}  % PDF/A-1b; requires <main>.xmpdata in project root
% \usepackage{lmodern}
% \input{glyphtounicode}
% \pdfgentounicode=1
% \usepackage{microtype}

%====================== TEMPORARY DEBUGGING (robust) ======================
% Hide all external images even if graphicx is already loaded:
\makeatletter
%\AtBeginDocument{\setkeys{Gin}{draft}}
\makeatother

% % OPTIONAL: also neutralize TikZ/pgfplots transparency during the test.
% % (Only applied if those packages are loaded)
% \makeatletter
% \@ifpackageloaded{tikz}{%
%   \tikzset{
%     opacity/.code      = {},
%     fill opacity/.code = {},
%     draw opacity/.code = {},
%     text opacity/.code = {}
%   }%
% }{}
% \@ifpackageloaded{pgfplots}{%
%   \pgfplotsset{
%     every axis plot/.append style={opacity=1, fill opacity=1},
%     every axis/.append style={opacity=1}
%   }%
% }{}
% \makeatother
%=========================================================================


%%%%%%%%%%%%%%%%%%%%%%%%%%%%%%%%%%%%%%%%%%%%%%%%%%%%%%%%%%%%%%%%%%%%%%%%%%%%%%%%%%%5
%%%%%%%%%%%%%%%%%%%%%%%%%%%%%%%%%%%%%%%%%%%%%%%%%%%%%%%%%%%%%%%%%%%%%%%%%%%%%%%%%

%====================== TEMPORARY DEBUGGING ================================
% put this once, anywhere after \documentclass
\PassOptionsToPackage{draft}{graphicx} % external images become empty boxes

%===================================================================


%------------ Geometry-----------------------
\usepackage[left=1.8cm,right=1.8cm,top=2cm,bottom=2.2cm]{geometry}
%\raggedbottom  %allow latex NOT to fill the page
\renewcommand{\baselinestretch}{1} %enforce single spacing between lines (standard)

%---------- Roman pages --------------
% set the page numbering to lowercase roman one
% for the contents and figures lists. It also resets
% page-numbering for the remainder of the dissertation (arabic, starting at 1)
\newenvironment{romanpages}
{\cleardoublepage\setcounter{page}{1}\renewcommand{\thepage}{\roman{page}}}
{\cleardoublepage\renewcommand{\thepage}{\arabic{page}}\setcounter{page}{1}}
\usepackage{titling}

%----------- Fonts and languages------
\usepackage[T1]{fontenc}
\usepackage[utf8]{inputenc} 
\usepackage[english]{babel} 
\usepackage{relsize}

%---------- Colors and links -------------------
\usepackage{xcolor}   
\usepackage{url} % for links
\usepackage{hyperref} % for links
\hypersetup{
    colorlinks=true,
    linkcolor=black,
    filecolor=magenta,
    citecolor = blue,
    urlcolor=cyan,
    %pdftitle={Overleaf Example},
    pdfpagemode=UseNone, % do not enforce fullpage view automatically
    }
\urlstyle{same}

%----------- New colors ------------------------ 
\definecolor{forestgreen}{rgb}{0.0, 0.5, 0.0}
\definecolor{niceviolet}{rgb}{0.5, 0.0, 0.6}
\definecolor{lightblue}{rgb}{0.4, 0.9, 1}
\definecolor{myyellow}{rgb}{0.9,0.75,0.0}


%----------------------- Color boxes for highlighting (adjust as you like) --------------------------------
\usepackage{tikz}
\usetikzlibrary{shadows}
\usetikzlibrary{angles,quotes}
\usepackage[most]{tcolorbox}

% separated box for highlighting
% usage \begin{tcolorbox}[fullbluebox] ... \end{tcolorbox}
\tcbset{
  fullbluebox/.style={
    enhanced,
    breakable,                    % allows multi-paragraphs
    parbox=false,                 % for indentation
    colframe=blue!70!black,
    colback=blue!5,
    boxrule=0.4pt,
    arc=2pt,
    boxsep=3pt,
    left=3pt,
    right=3pt,
    top=2pt,
    bottom=2pt,
    before skip=3pt,
    after skip=3pt,
    before upper={\setlength{\parindent}{15pt}},  % ensures indent works
  }
}

% separated box for highlighting
% usage \begin{tcolorbox}[fullredbox] ... \end{tcolorbox}
\tcbset{
  fullredbox/.style={
    enhanced,
    breakable,                    % allows multi-paragraphs
    parbox=false,                 % for indentation
    colframe=red!70!black,
    colback=red!5,
    boxrule=0.4pt,
    arc=2pt,
    boxsep=3pt,
    left=3pt,
    right=3pt,
    top=2pt,
    bottom=2pt,
    before skip=3pt,
    after skip=3pt,
    before upper={\setlength{\parindent}{15pt}},  % ensures indent works
  }
}

% inline box for highlighting
% use: \inlinefullbluebox{...your_content...}
\newtcbox{\inlinehighlightbox}{nobeforeafter, on line,
  colframe=blue!70!black,
  colback=blue!5,
  boxrule=0.4pt,       % thinner border
  arc=1pt,
  left=2pt, right=2pt, top=1pt, bottom=1pt
}

% separated box for highlighting
% usage \begin{tcolorbox}[emptybluebox] ... \end{tcolorbox}
\tcbset{
  emptybluebox/.style={
    colframe=blue!80!black,   % blue border
    colback=white,            % white background
    boxrule=0.8pt,            % border thickness
    sharp corners,            % no rounding
    enhanced,
    left=6pt, right=6pt, top=6pt, bottom=6pt
  }
}

% inline box for highlighting
% use: \inlineemptybluebox{...your_content...}
\newtcbox{\inlineemptybluenbox}{nobeforeafter, on line,
  colframe=blue!80!black,
  colback=white,
  boxrule=0.4pt,       % thinner border
  sharp corners,
  left=2pt, right=2pt, top=1pt, bottom=1pt
}


%==================================================================
%-------------------- MATHS, GRAPHS, IMAGES etc -------------------
%==================================================================

%----------- Math, symbol, graph and image packages ---------------
\usepackage[usestackEOL]{stackengine}
\usepackage{wrapfig}  % wrap text around figures
\usepackage{amsmath} 
\usepackage{amssymb} 
\usepackage{amsthm}
\usepackage{amsfonts}
\usepackage{mathtools}
\usepackage{mathrsfs}
\usepackage{esint}
\usepackage{braket} 
\usepackage{graphicx}
\usepackage{graphics}
\usepackage{pdfpages} %to display external pdf files as pages of the documents
\usepackage{tabularx} % for tables
\usepackage{booktabs} % for tables
\usepackage{stmaryrd}
\usepackage{textcomp}
\usepackage[export]{adjustbox}
\usepackage{cancel}
\usepackage{mathabx,epsfig}
\usepackage{cases}
\usepackage[overload]{empheq}
\usepackage{esvect}
\usepackage{nicefrac}
\usepackage{mathtools}
\usepackage[super]{nth}  
\usepackage[all]{xy}
\usepackage{subcaption} 
\usepackage{float}     % For more flexible float placement (e.g. [H] for here)
\usepackage{tikz}    
\usepackage[compat=1.1.0]{tikz-feynman} % For feynman diagrams
\tikzfeynmanset{warn luatex=false}
\usepackage{relsize}
\usepackage{stackengine}

\usepackage{tikz}
\usetikzlibrary{tikzmark,calc}



%--------------- Theorems, equations and proofs -------------------
\theoremstyle{definition}
\newtheorem{definition}{Definition}[section]
\newtheorem{theorem}[definition]{Theorem}
\newtheorem{lemma}[definition]{Lemma}
\newtheorem{corollary}[definition]{Corollary}
\newtheorem{proposition}[definition]{Proposition}
\newtheorem{fact}[definition]{Fact}
\newtheorem{example}[definition]{Example}
\newtheorem{claim}[definition]{Claim}
\newtheorem{remark}[definition]{Remark}
\newtheorem{conjecture}[definition]{Conjecture}
\newtheorem{notation}[definition]{Notation}
\newtheorem{assumption}[definition]{Assumption}
\newtheorem{objective}{Objective}
\newtheorem{open question}{Open question}
\newtheorem*{notation*}{Notation}
\newtheorem*{definition*}{Definition}
\newtheorem*{theorem*}{Theorem}
\newtheorem*{lemma*}{Lemma}
\newtheorem*{corollary*}{Corollary}
\newtheorem*{proposition*}{Proposition}
\newtheorem*{fact*}{Fact}
\newtheorem*{example*}{Example}
\newtheorem*{claim*}{Claim}
\newtheorem*{remark*}{Remark}
\newtheorem*{conjecture*}{Conjecture}
\newtheorem{conclusion}{Conclusion}
\newtheorem{consequence}{Consequence}

%------------------- equations and math displaying--------------------------
\numberwithin{equation}{section} % numbering reference for equations
\mathtoolsset{showonlyrefs=true,showmanualtags=true}  %only equations actually referred to are numbered
\allowdisplaybreaks[4] %set level of permissivity in breaking multiline equations across pages (4 is maximum)

%------------- Generic new math commands (adjust as needed) ------------- 
\newcommand{\N}{\mathbb{N}}
\newcommand{\Q}{\mathbb{Q}}
\newcommand{\R}{\mathbb{R}}
\newcommand{\T}{\mathbb{T}}
\newcommand{\Z}{\mathbb{Z}}
\newcommand{\C}{\mathbb{C}}
\newcommand{\E}{\mathbb{E}}
\DeclareMathOperator{\p}{\mathbb{P}} %probability symbol
\newcommand{\simsupset}{\raisebox{-0.8ex}{$\stackrel{\supset}{\sim}$}} %symbol for 'almost contained'



%--------------- Colored cancel lines to strike through terms----------
\newcommand{\redcancel}[1]{\renewcommand{\CancelColor}{\color{red}}\cancel{#1}}
\newcommand{\bluecancel}[1]{\renewcommand{\CancelColor}{\color{blue}}\cancel{#1}}
\newcommand{\greencancel}[1]{\renewcommand{\CancelColor}{\color{green}}\cancel{#1}}
\newcommand{\graycancel}[1]{\renewcommand{\CancelColor}{\color{gray}}\cancel{#1}}

%--------------- Paper specific new math commands (change each time)-------------------
\DeclareMathOperator{\pt}{s} % proper time
\DeclareMathOperator{\ct}{\tau} % conformal time 
\DeclareMathOperator{\fm}{\mathbf{f}_{m}} % growth rate
\DeclareMathOperator{\fsm}{\mathbf{f}_{s,m}} %growth rate for stable matter
\DeclareMathOperator{\fchim}{\mathbf{f}_{\chi,m}} %for decaying 
\DeclareMathOperator{\flm}{\mathbf{f}_{\ell,m}} %for decaying 
\newcommand{\m}{\mathrm{m}} % mother particle
\newcommand{\lcdm}{{\small$\Lambda$CDM }}
\newcommand{\gcdm}{{\small$\Gamma$CDM }}



%==================================================================
%-------------------- CODING -------------------
%==================================================================

\usepackage{listings} % for code
\usepackage{caption}  % better captions

\definecolor{codebg}{RGB}{248,248,248}
\definecolor{codeframe}{RGB}{200,200,200}

\lstdefinestyle{python}{
  language=Python,
  backgroundcolor=\color{codebg},
  frame=single,
  rulecolor=\color{codeframe},
  basicstyle=\ttfamily\small,
  keywordstyle=\color{blue},
  commentstyle=\color{gray},
  stringstyle=\color{orange},
  breaklines=true,
  showstringspaces=false,
  tabsize=4,
  captionpos=b
}

